\chapter*{Vorwort}\label{cha:vorwort}
Warum wird ein Draht heiss wenn Strom durchfliesst? Kann das mit dem Aufladen eines Handys verglichen werden? Ja, in beiden Fällen verursacht die Reibungsenergie der Elektronen die thermische Veränderung des Drahtes. Physikalisch gesehen ist der Fluss von geladenen Teilchen alltagsprachlich definiert als elektrischen Strom.

Stellen Sie sich vor, Sie untersuchen die kleinsten geladenen Teilchen, die man bisher kennt. Sie erhalten sehr genaue Messwerte und glauben, die Natur der Elementarladung erforscht zu haben. Sie veröffentlichen Ihre Arbeit, aber eifersüchtige Konkurrenten versuchen Sie niederzumachen, indem sie behaupten, Sie hätten Messwerte ausgeschlossen, die nicht stimmten.  Diesen Konkurrenten versuchen sie zu beweisen, dass das, was Sie erforscht haben, wahr ist. Es vergeht eine lange Zeit, bis Sie dies beweisen können. Etwa 13 Jahre später gewinnen Sie mit Ihrer Entdeckung der Elementarladung den Nobelpreis für Physik. Sie haben genau das durchgemacht, was Robert Andrews {\scshape Millikan} in den Jahren des Ersten Weltkrieges durchgemacht hat.