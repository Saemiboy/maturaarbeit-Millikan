\chapter{Reflexion}\label{cha:reflexion}
Im Allgemeinen konnte ich durch diese Arbeit meine Programmierkenntnisse vertiefen. Das Arbeiten mit \LaTeX~war anfangs herausfordernd, hat sich aber gelohnt, da es das Schreiben von Formeln vereinfacht hat und das Formatieren einer wissenschaftlichen Arbeit ebenfalls. 

Diese Arbeit hat gezeigt, wie es ist ein fortgeschritteneres Experiment im Fach Physik durchzuführen. Man hat gelernt, wie Messreihen aufgestellt werden, wie ein Experiment sorgfältig geplant werden muss und dass man nicht aufgeben soll, wenn das gewünschte Ergebnis nicht sofort eintritt. Das Experiment wird nicht beim ersten Versuch gelingen. Es erfordert Übung, Wissen und Planung, andernfalls wird es nicht gelingen. Eine weitere wichtige Erkenntnis war, dass es nicht schadet sich Unterstützung zu holen bei Fachpersonen und Mitschüler:innen. Das Experimentieren gelang erst als ein Assistenten hinzugezogen wurde. \\

\noindent Die vorliegende Arbeit hat zudem gezeigt, dass eine sorgfältige Arbeitsweise sich auszahlen kann und dass die Inanspruchnahme externer Hilfe bei Schwierigkeiten zu einem erfolgreichen Ergebnis führen kann.

