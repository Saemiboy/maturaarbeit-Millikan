\chapter{Einleitung}
Im nächsten Abschnitt wird zuerst auf die Ausgangslage und Problemstellung eingegangen. Anschliessend wird die Fragestellung und Zielsetzung der Untersuchung erläutert, geflogt von der Abgrenzung des Themas. Im letzten Abschnitt der Einleitung wird der Aufbau behandelt.

\section{Ausgangslage und Problemstellung}\label{sec:ausgangslage}
Die Elementarladung, eine der wichtigsten Konstanten der Physik, beschreibt die kleinste elektrische Ladung, die in der Natur vorkommt. Der Literaturwert der Elementarladung beträgt, $1.602176634 \cdot 10^{-19} C$ \parencite[123]{fundamentum_mathe}. Sie bildet die Grundlage für eine Vielzahl physikalischer Phänomene sowie technischer Anwendungen. Ihre präzise Bestimmung stellt einen signifikanten Meilenstein in der Geschichte der Wissenschaft dar. Robert Millikan führte 1909 mit seinem Öltröpfchenexperiment den ersten erfolgreichen Nachweis der Elementarladung durch und bewies, dass elektrische Ladungen nur in diskreten Vielfachen vorkommen \parencite{PhysRevSeriesI.32.349}

Obwohl die experimentelle Bestimmung der Elementarladung von zentraler Bedeutung ist, stellt sie eine Herausforderung dar. Da die Ladung eines Elektrons äusserst klein ist, erfordert das Experiment präzise Messtechniken und eine sorgfältige Kontrolle von den unterschiedlichen Messgrössen. Bereits zur Zeit in der das Experiment entwickelt wurde,  gab es Kritik an der Genauigkeit seiner Messungen, was zeigt, wie anspruchsvoll diese Art von Forschung war und immernoch ist \parencite{dewiki:247237013}.

\section{Zielsetzung und Fragestellung}\label{sec:zielsetzung}
Ziel dieser Arbeit ist es, die Genauigkeit des Ergebnisses einer Messung der Elementarladung zu untersuchen, wenn das Experiment, das vor mehr als hundert Jahren entwickelt wurde, mit modernen Mitteln wiederholt wird. Die Genauigkeit des Ergebnisses ist dabei mit Hilfe einer Fehlerrechnung zu bestimmen.

Die Zentrale Fragestellung dieser Arbeit lautet daher: \glqq Wie genau lässt sich die Elementarladung mit der Wiederholung des historischen Öltröpfchen Experiment nach Millikan unter modernen Bedingungen zu bestimmen? \grqq 

\section{Abgrenzung}
Die Arbeit konzentriert sich ausschliesslich auf die experimentelle Bestimmung der Elementarladung andhand des Öltröpfchenexperiments. Andere Methoden zur Bestimmung der Elementarladung, wie das Thomson Experiment oder die Elektrolyse, werden nur am Rande erwähnt. Die Relevanz der Elementarladung wird anhand praktischen Beispiele aus der Wissenschat und Technik dargestellt, ohne dabei auf eine vollständige Abdeckung aller Anwendungsgebiete abzuzielen.

\section{Aufbau}
Dieser Abschnitt beschreibt die grundlegende Struktur der Arbeit. Ziel dabei ist es, eine klare Orientierung durch die einzelnen Kapitel zu schaffen. Zunächst wird die Einleitung präsentiert. Sie soll die Ausgangslage und Problemstellung (\autoref{sec:ausgangslage}) und Zielsetzung (\autoref{sec:zielsetzung}) aufzeigen. Anschliessend führen die theoretischen Grundlagen (\autoref{ch:theorieGrundlagen}) in das Thema ein, bevor der experimentelle Aufbau (\autoref{cha:experimentAufbau}) beschrieben wird.

Die Durchführung des Experiments (\autoref{cha:durchfuehrung}) wird in verschiedenen Schritten gegliedert. Zu den einzelnen Schritten gehören zum Beispiel das Ausrichten des optischen Systems (\autoref{sec:optischesSystem}) oder die Steuerungsfunktionen der einzelnen Schalter (\autoref{sec:funktionen}). Die eigentliche Versuchsdurchführung wir dann in \autoref{sec:durchfuehrung} beschrieben. Die gesammelten Daten werden anschliessend in der Auswertung (\autoref{cha:auswertung}) analysiert und auf Genauigkeit überprüft.

Der Abschluss der Arbeit bildet das Kapitel Diskussion (\autoref{cha:fazit}), in dem die Messergebnisse diskutiert werden.