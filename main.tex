\documentclass[a4paper,11pt]{report}


\usepackage[german]{babel}        % Deutschsprachige Beschriftungen
\usepackage[utf8]{inputenc}       % Utf8 Zeichensatz
\usepackage[T1]{fontenc}          % Schriftenkodierung
\usepackage[lighttt]{lmodern}     % Schriftart
\usepackage{amsmath}  
\usepackage{amsfonts}            % Mathematische Formeln
\usepackage[normalem]{ulem}       % Durchgestrichener Text
\usepackage{xcolor}               % Farbiger Text
\usepackage{verbatim}             % Text ohne Formatierung
\usepackage{listings}             % Code mit Formatierung
\usepackage{csquotes}             % Kontextsensitive Zitatanlage
\usepackage{caption}              % Erweiterte Beschriftungen
\usepackage{subcaption}           % Unterbeschriftungen
\usepackage{geometry}             % Seitenränder
\usepackage{setspace}             % Zeilenabstand
\usepackage{fancyhdr}             % Header und Footer
\usepackage{graphicx}             % Grafiken
\usepackage{wrapfig}
\usepackage{float}
\usepackage{longtable}	
\usepackage{svg}                  % SVG Grafiken
\usepackage{booktabs}             % Tabellen
\usepackage{tabularx}             % Breite von Boxen
\usepackage[backend=biber, style=apa]{biblatex} % Referenzen
\usepackage[hidelinks]{hyperref}             % Hyperlinks im PDF-Dokument
\usepackage{hyperxmp}             % Metadaten im PDF-Dokument
\usepackage{makeidx}              % Optional, für Glossar (Index)
\usepackage{siunitx}   

\title{Die experimentelle Bestimmung der Elementarladung}
\date{\today}
\author{Samuel Egli}
\makeatletter            % Erlaubt das Auslesen der obigen Eigenschaften
\let\papertitle\@title   % Titel in \papertitle speichern
\let\paperdate\@date     % Datum in \paperdate speichern
\let\paperauthor\@author % Autor in \paperauthor speichern
\makeatother
\def\paperinstitution{Kantonsschule am Burggraben}
\def\papertype{Maturaarbeit}
\def\papersupervisor{Dr. Rheinhard Gross}

% ======== START VON SEITENRÄNDER ========= (Seitenränder)
\geometry{
	a4paper,
	total={150mm,237mm},
	left=30mm,
	top=30mm,
}
\setlength{\headheight}{14pt}
% ========= ENDE VON SEITENRÄNDER =========


\addbibresource{literature.bib}

\begin{document}
	
\begin{titlepage}
	\centering
	\includegraphics[width=.3\textwidth]{title_logo.pdf}\\[.25cm]
	{\scshape\paperinstitution\par}
	\vspace{1cm}
	{\scshape\Large\papertype\par}
	\vspace{1.5cm}
	{\huge\bfseries\papertitle\par}
	\vspace{2cm}
	{\Large Vorgelegt durch:\par\paperauthor\par}
	\vfill
	\includegraphics{titelbildMaturaarbeit.pdf}
	\vfill
	Vorgelegt bei:\par
	{\sc\papersupervisor\par}
	\vspace{1cm}
	{\large\paperdate\par}
\end{titlepage}
	
	% === INHALTSVERZEICHNIS ===
	\tableofcontents %          \
	\newpage %                  \
	% --------------------------
	
	% =========== START VON STYLING ===========
	\onehalfspacing % Zeilenabstand 1.5
	\renewcommand{\chaptermark}[1]{\markboth{\MakeUppercase{\thechapter.\ #1}}{}} % Kapitel
	\renewcommand{\sectionmark}[1]{\markright{\thesection.\ #1}{}} % Abschnitt
	\fancyhead[R]{\rightmark}
	\fancyhead[L]{\leftmark}
%	\fancyhead[C]{\thepage{}} % Seitennummer
	\fancyhead[L]{\leftmark}
	\fancyhead[R]{\rightmark}
	\fancyfoot[C]{\thepage{}}
	\pagestyle{fancy}
	% =========== ENDE VON STYLING ============
	
	\chapter{Einführung}\label{ch:einfuerung}
\section{Einleitung}\label{sec:einleitung}
Warum wird ein Draht heiss wenn Strom durchfliesst? Kann das mit dem Aufladen eines Handys verglichen werden? Ja, in beiden Fällen verursacht die Reibungsenergie der Elektronen die thermische Veränderung des Drahtes. Physikalisch gesehen ist der Fluss von geladenen Teilchen alltagsprachlich definiert als elektrischen Strom.
  
Stellen Sie sich vor, Sie untersuchen die kleinsten geladenen Teilchen, die man bisher kennt. Sie erhalten sehr genaue Messwerte und glauben, die Natur der Elementarladung erforscht zu haben. Sie veröffentlichen Ihre Arbeit, aber eifersüchtige Konkurrenten versuchen Sie niederzumachen, indem sie behaupten, Sie hätten Messwerte ausgeschlossen, die nicht stimmten.  Diesen Konkurrenten versuchen sie zu beweisen, dass das, was Sie erforscht haben, wahr ist. Es vergeht eine lange Zeit, bis Sie dies beweisen können. Etwa 13 Jahre später gewinnen Sie mit Ihrer Entdeckung der Elementarladung den Nobelpreis für Physik. Sie haben genau das durchgemacht, was Robert Andrews {\scshape Millikan} in den Jahren des Ersten Weltkrieges durchgemacht hat.

Diese Arbeit soll zeigen, wie man zu Beginn des 20. Jahrhunderts ein solches Experiment erfunden hat, wie man vor mehr als hundert Jahren so genaue Messwerte erhalten hat und wie genau solche Messungen sind, wenn man das Experiment heute durchführt.

% Quelle: https://de.wikipedia.org/wiki/Robert_Andrews_Millikan

\section{Biographie von Millikan}\label{sec:autobiographie}
Robert Andrews {\sc Millikan} wurde 1868 in Amerika geboren. Im Alter von 18 Jahren begann er am Oberlin College in Ohio zu studieren. Zunächst studierte er Mathematik und Griechisch, später belegte er einen Kurs in Physik und legte sein Examen als Physiklehrer ab. Etwa 10 Jahre später promovierte er an der Columbia University. Nach seiner Promotion ging er für ein Jahr nach Deutschland, um seine Kenntnisse bei Max {\sc Planck} und Walther {\sc Nernst} zu vertiefen. Danach kehrte er in die USA zurück, wo er 10 Jahre als Professor an der University of Chicago arbeitete. 

1909 begann er, die Natur der Elementarladung zu erforschen. Anfangs benutzte er die \textit{Tröpfchenmethode}, die mit Wasser durchgeführt wurde. Später benutzte er die \textit{Öltröpfchenmethode}, die für die Bestimmung der Elementarladung besser geeignet war, da sich Öltröpfchen im Vergleich zu Wassertröpfchen als stabiler erwiesen.  Mit dieser Methode gelang es ihm, die Einheit der kleinsten elektrischen Ladung zu bestimmen, die er mit $e$ bezeichnete. Ein Jahr später veröffentlichte er seine Arbeit mit mehr als 38 Messungen. Sie stieß bei anderen Forschern auf grosses Interesse, aber auch auf heftige Kritik. Um die Kritik zu entkräften, veröffentlichte er drei Jahre später eine weitere Arbeit über die experimentelle Bestimmung der Elementarladung, doch auch diese Ergebnisse wurden angezweifelt. In den Jahren vor dem Ersten Weltkrieg erhielt er 3-4 Auszeichnungen, darunter den Comstock-Preis für Physik.

Millikan untersuchte nicht nur die Natur der Elementarladung, sondern wollte auch die Lichtquantenhypothese von Albert {\sc Einstein} experimentell überprüfen, da er Einsteins Interpretation skeptisch gegenüberstand. Es gelang ihm jedoch, die Richtigkeit von Einsteins Gleichungen zu beweisen.

Als Millikan 1918 sein Buch "Das Elektron" veröffentlichte, behauptete er, seine Messungen der Elementarladung seien genauer als die seiner Konkurrenten, da die Werte nur wenig streuten. Diese Arbeit begründete seinen späteren Ruhm und die Verleihung des Nobelpreises im Jahr 1923. 

In der Zwischenkriegszeit setzte er seine Forschungen fort, bis er 1946 in den Ruhestand trat. Er schrieb zahlreiche Bücher über Natur und Religion sowie verschiedene Lehrbücher. %\parencite[vgl.][Millikan]{dewiki:247237013}

\section{Relevanz der Elementarladung Heute}\label{sec:relevanz}
In welchen Bereichen des täglichen Lebens benötigen wir heute Elementarladungen? Die wohl bekannteste Technik, bei der wir reine Elementarladungen (Elektronen) benötigen, ist die Röntgentechnik in der Medizin. Ein Röntgengerät ist nichts anderes als ein Teilchenbeschleuniger, der Elektronen auf den bzw. durch den Körper schiesst. Ein anderes Beispiel aus der Medizin ist das Bestrahlungsgerät in der Krebstherapie. Hier werden keine Elektronen, sondern Protonen mit genau der gleichen Ladung, $1.602176634 \cdot 10^{-19} C$ \parencite[123]{fundamentum_mathe}, aber positiv statt negativ, beschleunigt und auf den Körper geschossen. 

Ohne das Wissen, dass es keine Ladung gibt, die kleiner als die Elementarladung ist, würde heute keines unserer elektronischen Geräte, insbesondere keine elektronischen Rechner, funktionieren. Denn jedes Bit in unseren Chips basiert darauf, ob ein Elektron fehlt oder nicht.

Um die Masse eines Elektrons zu bestimmen, benötigt man auch die Elementarladung \textit{e}. Mit Hilfe eines Magneten wird ein Elektron auf eine Kreisbahn geschickt. Dabei wirkt auf das Elektron eine magnetische Kraft (Lorentzkraft). Um eine Kreisbewegung zu erzielen, braucht es eine Zentripetalkraft, die von der Lorentzkraft aufgebracht wird. Formal ausgedrückt bedeutet das: $F_L \ = \ F_Z$, wobei $F_L \ = \ e \cdot v \cdot B$. Dabei steht $e$ für die Elementarladung

   

	\chapter{Theoretische Grundlagen}\label{ch:theorieGrundlagen}
\section{Elementarladung}\label{sec:elementarladung}
\subsection{Definition}\label{sub:definition}
Die Elementarladung wird physikalisch definiert als,

\begin{equation}\label{eq:definition}
 q  =  n \cdot e \:  \Leftrightarrow \: e = \frac{q}{n} \quad | \ n \in \mathbb{Z}
\end{equation}

\noindent Diese Definition bedeutet nichts anderes, als dass alle möglichen Ladungen ganzzahlige Vielfache der Elementarladung $e$ sind. Diese Erkenntnis bekommt man über den Millikan-Versuch, der aufzeigt, dass sich die Ladungen von Körpern nicht kontinuierlich verteilen, sondern nur in Etagen vorkommen. 

Die Elementarladung besitzt die Einheit Coulomb C. Sie steht als Einheitssymbol für die physikalische Grösse der Ladung $[Q]$. Manchmal wird anstatt Coulomb auch die alternative Schreibweise, die Amperesekunde, verwendet. Das soll Sie aber nicht verwirren, denn die Einheit Coulomb setzt sich aus dem Ampere $[I]$ und der Zeit $[t]$ zusammen. Formal ausgedrückt bedeutet dass: $I \cdot t = Q$. 

\subsection{Eigenschaften}\label{sub:eigenschaften}
Wie oben in \autoref{sec:elementarladung} hergeleitet, hat die Elementarladung die Eigenschaft, dass sie die kleinst mögliche Ladungseinheit in der Natur ist. Kontrovers wird es wenn man Ihnen jetzt sagt das auch drittel Elementarladungen möglich sind. Ein Proton, das die Ladung 1e beträgt, besteht aus drei kleinsten Elementarteilchen, den Quarks. Quarks sind die kleinsten im Moment bekannten Elementarteilchen und sie sind die Bausteine der Materie. Es gibt verschiedene Arten von Quarks, wir beschäftigen uns aber nur mit den Up und Down Quarks. Das Proton besteht aus zwei Up-Quarks und einem Down-Quark. in der folgenden \autoref{tab:quark_tabelle} kann man die verschiedenen Ladungen der Quarks sehen. Wenn man diese Ladungen nun zusammenrechnet kommt man wieder auf die Elementarladung e. 

\begin{equation}\label{eq:mathematische_zusammensetzung_von_qurks}
	2 \cdot \left(\frac{2}{3}e\right)  + 1 \cdot \left( -\frac{1}{3}e\right) 
	= \frac{4}{3}e - \frac{1}{3}e = 1e
\end{equation}

\noindent Wie in \autoref{eq:mathematische_zusammensetzung_von_qurks} gezeigt, kann die Elementarladung auch aus Bruchteilen von sich selber bestehen. Wieso hat man jetzt genau die Ladung eines Elektron oder Proton als Elementarladung festgelegt? Das ist sehr einfach zu beantworten, wenn man die Verhaltensweise von Quarks kennt. Quarks kommen nie einzeln in der Natur von, sondern nur in sogenannten Quark-Gluon-Plasmen. Einfach ausgedrückt nur als Pärchen in einem Proton oder Neutron.

%\begin{figure}[h]\label{fig:quarkTabelle}
	%\begin{center}
		%\includegraphics[scale=1]{bilder/pdf/quark_tabelle.pdf}
	%	\caption{Tabelle Ladung von Quarks}
%	\end{center}
%\end{figure}

\begin{table}[ht]
	\begin{center}
		\begin{tabular}{l|l}
			\multicolumn{2}{c}{\textit{\textbf{Quark Tabelle}}} \\
			\hline
			Art & Ladung \\
			\hline
			Up u & $+ \frac{2}{3}e$ \\
			\hline
			Down d & $- \frac{1}{3}e$\\
			\hline
		\end{tabular}
	\end{center}
	\caption{Up-Down-Quark Ladungen}
	\label{tab:quark_tabelle}
\end{table}

\section[Historische Methoden]{Historische Methoden zur Bestimmung der Elementarladung}
\subsection{Thomsonsche Methode}\label{sub:thomson}
In dieser Arbeit geht es hauptsächlich um den Millikan-Versuch, zur Bestimmung der Elementarladung. Andere Methoden sind aber dennoch nennenswert. Zum Beispiel das Thomsonsche Experiment. Dabei geht es um einen Versuch, der ein Elektronenstrahl durch ein magnetisches Feld schiesst. Dabei wird der Strahl durch die Lorenzkraft abgelenkt und Joseph John {\scshape Thomson} konnte, durch verändern des Magnetfeldes, das Verhältnis von Masse und Ladung $\frac{e}{m}$ entdecken. Durch dieses Verhältnis konnte er die Elementarladung noch nicht bestimmen. Erst später als man die Masse eines Elektron entdeckte, konnte man indirekt über dieses Verhältnis auf die Ladung zurückschliessen. 

\subsection{Elektrolyse}\label{sub:elektrolyse}
Eine andere Methode, die Elementarladung zu ermitteln, funktioniert mithilfe der Elektrolyse. Bei der Elektrolyse wird eine Spannung angelegt um chemische Reaktionen (zum Beispiel Zersetzung von Molekülen) in einer ionischen Lösung zu erzwingen. Dabei kann man durch Messungen der Spannung und Anzahl Ionen, die gewandert sind, auf die Elementarladung schliessen. 

Mit beiden dieser Methoden kann man die Elementarladung indirekt bestimmen. Sie sind für diese Arbeit sicher nennenswert, jedoch kann man mithilfe des Millikan Versuchs viel direkter Ladungen kleinster Partikel messen.

\section{Theorie des Versuchs}\label{sec:versuchsTheorie}
Da wir die anderen Methoden nur kurz angeschnitten haben, wird der Millikan Versuch jetzt genauer erklärt oder zumindest die Theorie dahinter. 

Wir beginnen mit einem Öltröpfchen im Freien Fall. Folgende \autoref{fig:freierFall} ist dazu zu sehen.

\begin{figure}[h]
	\begin{center}
		\includegraphics[scale=0.5]{bilder/pdf/Abbildung1_FreierFall.pdf}
		\caption{Öltröpfchen im Freien Fall}
		\label{fig:freierFall}
	\end{center}
\end{figure}

\noindent In \autoref{fig:freierFall} sehen wir welche Kräfte auf ein Öltröpfchen im freien Fall wirken. Nach unten haben wir die Gewichtskraft, die Abhängig ist von der Masse \textbf{$m$} und dem Ortsfaktor \textbf{$g$}. Das Öltröpfchen fällt in der Luft und hat seine Endgeschwindigkeit erreicht (die dazu benötigte Zeit beträgt wenige Millisekunden). Die Kraft, die nach oben zeigt, ist die Reibungskraft der Luft. Sie ist abhängig von der Fall- bzw. Endgeschwindigkeit \textbf{$v_f$} und dem Reibungskoeffizienten \textbf{$k$} von Luft und dem Tröpfchen.
Diese Kräfte sind genau gleich gross, weil sie in einem Kräftegleichgewicht sind. 

\begin{equation}\label{eq:kräfteFreierFall}
	mg \ = \ kv_f
\end{equation}   

\noindent Nun setzen wir dieses Öltröpfchen in ein elektrisches Feld. Mit den Kräftevektoren eingezeichnet, sieht das so aus.

\begin{figure}[h]
	\begin{center}
		\includegraphics[scale=0.5]{bilder/pdf/Abbildung2_elektrischesFeld.pdf}
		\caption{Öltröpfchen im elektrischen Feld}
		\label{fig:elektrischesFeld}
	\end{center}
\end{figure}

\noindent Die elektrische Kraft, die in \autoref{fig:elektrischesFeld} nach oben zeigt, ist abhängig von der elektrischen Feldstärke $E$ und der Ladung $q$ des Tröpfchen. Da die elektrische Kraft nun grösser als die Gewichtskraft ist, steigt das Tröpfchen. Wie wir schon oben im Freien Fall behandelt haben, gibt es wieder eine Reibungskraft der Luft die entgegengesetzt der Bewegungsrichtung verläuft. Dieses Mal ist sie aber nicht von der Fallgeschwindigkeit abhängig, sondern von der Steiggeschwindigkeit $v_r$ (steig auf Englisch: rise) und wie oben von dem Reibungskoeffizienten der Luft $k$. Wenn man jetzt diese Vektoren algebraisch addiert, kommt man auf folgende Gleichung.

\begin{equation}\label{eq:elektrischesFeld}
	Eq \ = \ mg \cdot kv_r
\end{equation}

\noindent Nun kann man nach $q$ umstellen und den Reibungskoeffizienten $k$ mithilfe der \autoref{eq:kräfteFreierFall} eliminieren. 

\begin{equation}\label{eq:hauptgleichung}
	q \ = \ \frac{mg \cdot (v_f + v_r)}{Ev_f}
\end{equation}

\noindent Die Masse eines Öltröpfchens zu bestimmen, ist in diesem Fall fast unmöglich. Aus diesem Grund versucht man über die Dichte des Öls $\rho$, und das Volumen der Ölkugel, auf die Masse zu kommen. Der Zusammenhang von Dichte und Masse sieht folgendermassen aus: $\rho \ = \ \frac{m}{V} \ \Leftrightarrow \ m \ = \ \rho \cdot V$. Das Volumen kann jetzt noch ausgerechnet werden mithilfe des Radius $a$. Setzt man nun alles zusammen, kommt man auf folgende Formel für die Masse eines Öltröpfchens. 

\begin{equation}\label{eq:masseFormel}
	mg \ = \ \frac{4}{3} \pi a^3 \rho g
\end{equation}

\noindent Wir können jetzt dieses $m$ mit dem $m$ in \autoref{eq:hauptgleichung} substituieren.

\begin{equation}\label{eq:ladungFormel}
	q \ = \ \frac{4\pi a^3\rho g (v_f + v_r)}{3(Ev_f)}
\end{equation}

\noindent Das neue Problem wird jetzt der Radius $a$ sein. Die Tröpchen sind zu klein, um den Radius zu messen. Die Lösung des Problems finden wir im stokesschen Reibungsgesetz $(F_f \ = \ 6\pi \eta a v_f)$. Es zeigt den Zusammenhang von Fallgeschwindigkeit und Reibungskraft der Luft. Diese Formel beschreibt, wie sich ein Kugelförmiges Objekt in einem viskosem Medium verhält. Dieses Gesetz hängt von der Reibungszahl der Luft $\eta$ und der Fallgeschwindigkeit $v_f$ ab. Wir können diesen Ausdruck mit dem rechten Ausdruck von Gleichung \ref{eq:masseFormel} gleichsetzen. Wenn man nach $a$ auflöst erhält man:

\begin{equation}\label{eq:stokesRadius}
	a \ = \ \sqrt{\frac{9\eta v_f}{2\rho g}}
\end{equation}

\noindent Das stokessche Reibungsgesetz wird leider inkorrekt wenn die Fallgeschwindigkeit weniger als 0.1 cm/s beträgt. Da wir es im Experiment mit Fallgeschwindigkeiten zwischen 0.01 und 0.001 cm/s (zwischen $10^-4$ und $10^-6$ m/s) zu tun haben, müssen wir das Reibungsgesetz mit einem Korrekturfaktor multiplizieren. Die effektive Viskosität resultiert aus:

\begin{equation}\label{eq:effViskosität}
	\eta_{eff} \ = \ \eta \left( \frac{1}{1 + \frac{b}{pa}} \right) 
\end{equation}

\noindent $b$ ist dabei eine Konstante und $p$ ist der atmosphärische Druck in Pascal. 

\noindent Nun wird $\eta_{eff}$ in Gleichung \ref{eq:effViskosität} für $\eta$ in Gleichung \ref{eq:stokesRadius} substituiert. 

\begin{equation}\label{eq:korrekturRadius}
	a \ = \ \sqrt{\frac{9\eta v_f}{2\rho g} \left( \frac{1}{1 + \frac{b}{pa}}\right)}
\end{equation}

\noindent \autoref{eq:effViskosität} enthält den Radius $a$. Das Problem ist, dass wir einen Term für $a$ gefunden haben, der $a$ aber enthält. Der Ausdruck für $a$ in \autoref{eq:korrekturRadius} kann in eine quadratische Gleichung umgewandelt werden:

\begin{equation}\label{eq:quadraticRadius}
	\begin{split}
		a & \ = \ \sqrt{\frac{9\eta v_f}{2\rho g} \left( \frac{1}{1 + \frac{b}{pa}}\right)} \\
		a^2 & \ = \ \frac{9\eta v_f}{2\rho g} \left( \frac{1}{1 + \frac{b}{pa}}\right) \\
		a^2 + \frac{b}{p}a & \ = \ \frac{9\eta v_f}{2\rho g} \\
		a^2 + \frac{b}{p}a - \frac{9\eta v_f}{2\rho g} & \ = \ 0
	\end{split}
\end{equation} 

\noindent Jetzt wird \autoref{eq:quadraticRadius} nach $a$ aufgelöst:

\begin{equation}\label{eq:qRadius}
	a \ = \ \sqrt{\left( \frac{b}{2p}\right)^2 + \frac{9\eta v_f}{2\rho g}} - \frac{b}{2p}
\end{equation}

\noindent Es ist zu beachten, dass, nicht wie bei \autoref{eq:korrekturRadius}, jetzt kein $a$ im Ausdruck mehr vorkommt. Jetzt wird der komplette Term für $a$ in Gleichung \ref{eq:ladungFormel} ersetzt.

\begin{equation}\label{eq:ladungMitEingesetzR}
	q \ = \ \frac{4\pi \left[ \sqrt{\left( \frac{b}{2p}\right)^2 + \frac{9\eta v_f}{2\rho g}} - \frac{b}{2p} \right]^3 \rho g(v_f + v_r) }{3(Ev_f)}
\end{equation}

\noindent Die Elektrische Feldstärke $E$ kann auch so ausgedrückt werden:

\begin{equation}\label{eq:elektrischeFeldstärke}
	E \ = \ \frac{V}{d}
\end{equation}

\noindent Wenn jetzt $E$ aus Gleichung \ref{eq:ladungMitEingesetzR} mit $E$ aus \autoref{eq:elektrischeFeldstärke} ersetzt wird und die ganze Gleichung noch schöner umgeformt wird, resultiert daraus:

\begin{equation}\label{eq:letzteFormel}
	q \ = \ \frac{4\pi}{3} \cdot \left[ \sqrt{\left( \frac{b}{2p}\right)^2 + \frac{9\eta v_f}{2\rho g}} - \frac{b}{2p} \right]^3 \cdot \frac{\rho gd(v_f + v_r)}{Vv_f}
\end{equation}

\noindent Die Quelle für all diese Berechnungen basieren auf \parencite{instructionManual}




	


	\chapter{Experimenteller Aufbau}\label{cha:experimentAufbau}
\section{Versuchsanordnung}\label{sec:versuchsanordnung}

Wie schon in \autoref{sec:versuchsTheorie} besprochen, beruht das Millikan-Experiment auf dem Kräftegleichgewicht von Gewichtskraft und elektrischer Kraft. Zuerst wird ein dunkler Raum gebraucht. Am Besten funktioniert es mit einer Dunkelkammer, in der kein Licht ist. Das einzige Licht, das gebraucht wird ist Mikroskoplicht am Experimentapparat. Es werden während dem Experiment sehr kleine Öltröpchen mit einem Zerstäuber in eine Kammer gesprüht. Danach wird anhand des Lichtes und dem Mikroskop die Fallgeschwindigkeit des Tröpfchens gemessen. Der Boden und die Decke der Kammer, bestehen aus elektrischen Kapazitoren. Das bedeutet, die Kammer kann ein elektrisches Feld erzeugen. Mit einem Schalter kann die Richtung des elektrischen Feldes gewechselt werden. Diese Funktion wird bei der zweiten Messung gebraucht. Da werden die Kapazitoren eingeschaltet, so dass das elektrische Feld nach oben zeigt (Decke + Boden -). Wenn die Tröpfchen negativ geladen sind, werden sie die Schwerkraft überwinden können und werden nach oben steigen. Dabei wird wieder die Geschwindigkeit gemessen, die die Tröpchen brauchen, um von einer Linie des Gitters zur anderen zu kommen. Diese ganze Prozedur wiederholt man, bis das Tröpfchen nicht mehr gesehen werden kann. Eine schöne Schritt-für-Schritt Anleitung wird in \autoref{sec:durchfuehrung} gezeigt.

\section{Material}\label{sec:material}

In dieser Arbeit wurde das \textit{Model AP-8210 von PASCO scientific} mit der Halogenlampe verwendet. \\

\noindent \textbf{Material, das dabei ist:}

\begin{itemize}
	\item Apparat Plattform und Kondensator Ladungsschalter (Eine genauere Beschreibung der Plattform in \autoref{sub:inhaltApparatur})
	\item 12 Volt DC Transformator für die Halogen Lampe
	\item nicht flüchtiges Öl
	\item Ölsprüher 
\end{itemize}

\subsection{Plattform}\label{sub:inhaltApparatur}
Da das Experiment schon fertig gebaut ist, werden jetzt alle Komponenten aufgezählt, die sich auf der Plattform befinden.

\noindent \textbf{Komponenten Plattform:}

\begin{itemize}\label{item:apparatur}
	\item Tröpfchenbetrachtungskammer (Wird im nächsten \autoref{sub:viewingChamber})
	\item Betrachtungsfernrohr (30X, Hellfeld, aufrechtes Bild) mit Fadenkreuz (Linienabstand: 0,5 mm große Teilung, 0,1 mm kleine Teilung), Fadenkreuz-Fokussierring und Tropfenfokussierring
	\item Halogen Lampe (12 V, 5 W)
	\item Fokussierdraht
	\item Kondensatorenspannungs Anschlüsse
	\item Thermistor Anschlüsse (sind an den unteren Kondensator eingebaut)
	\item Thermistor Tabelle (Widerstand-Temperatur)
	\item Ionisationsquellen Schalter (3 verschiedene Positionen: Ionisation AN, Ionisation AUS, Sprüh Position)
	\item Wasserwaage
	\item Kondensator Ladungsschalter (mit einem Meter Kabel, um Vibrationen aus dem Weg zu gehen)
\end{itemize}


\subsection{Betrachtungskammer}\label{sub:viewingChamber}
Die Betrachtungskammer kann auseinandergenommen werden. Die einzelnen Komponenten werden hier aufgelistet.

\noindent \textbf{Einzelteile der Betrachtungskammer:}

\begin{itemize}\label{item:betrachtungskammer}
	\item Deckel
	\item Gehäuse
	\item Tröpfchenlochabdeckung
	\item obere Kondensatorplatte
	\item Abstandshalter aus Plastik (ungefähr 7.6mm dick)
	\item untere Kondensatorplatte
	\begin{itemize}
		\item Thorium-232 Alphateilchenquelle
		\item elektronische Verbindung zur oberen Platte
	\end{itemize}
	\item konvexe Linse
\end{itemize}

\begin{figure}[h]
	\centering
	\begin{minipage}[t]{0.45\textwidth}
		\centering
		\includegraphics[scale=0.5]{bilder/pdf/plattformKomponenten.pdf}
		\caption{Komponenten der Plattform}
		\label{fig:plattformKomp}
	\end{minipage}
	\hfill
	\begin{minipage}[t]{0.45\textwidth}
		\centering
		\includegraphics[width=\textwidth]{bilder/pdf/BetrachtungsKammerKomponenten.pdf}
		\caption{Komponenten der Betrachtungskammer}
		\label{fig:betrachtKomp}
	\end{minipage}
\end{figure}






	\chapter{Durchführung}\label{cha:durchfuehrung}
\section{Vorbereitung}\label{sec:vorbereitung}
\subsection{Auswahl der Umgebung und Höhe}\label{sub:auswahlUmgebung}
Um das Experiment durchzuführen müssen verschiedene Vorbereitungen vorgenommen werden. Zuerst muss der Ort ausgewählt werden. Um das Experiment möglichst erfolgreich und präzise zu halten, sollte es in einem möglichst dunklen Raum durchgeführt werden. Für diese Arbeit wurde die Dunkelkammer (Zimmer G10) der Kantonsschule am Burggraben zur Verfügung gestellt. Ein weiterer Punkt der Vorbereitung ist der Untergrund, auf dem das Experiment steht. Im Experimentierkasten kommen Verlängerungsstäbe mit, die dazu dienen das Experimentieren angenehmer zu machen. Das Problem mit diesen Stäben ist, dass sie Vibrationen nicht weghalten und das Experiment sehr anfällig für diese ist. In dieser Arbeit wurde die Plattform auf ein Holzklotz gelegt, damit das Experiment auf Augenhöhe ist bei gestrecktem Rücken. Die Plattform sollte gerade stehen. Das kann man mit der Wasserwaage auf der Plattform kontrollieren. Falls die Plattform schräg sein sollte, kann man mit den veränderbaren Füssen die Plattform ausebenen. Mit diesem Holzklotz hatte das Experiment einen festen Untergrund und war somit bereit für das Einstellen des optischen System. 

\subsection{Kondensatorenabstand messen}\label{sub:kondensatorenabstand}
Der nächste Schritt für die Vorbereitung ist das Messen des Abstandes zwischen den beiden Kondensatoren. Das Wichtige bei diesem Schritt ist, dass die Spannung abgeschaltet ist. Zuerst wird das Gehäuse der Betrachtungskammer abgenommen. Danach wird die obere Platte vorsichtig weggenommen und die darunter liegende Platte aus Kunststoff auch. Im Experimentierkasten befindet sich eine Schieblehre. Damit misst man die Dicke der Kunststoffplatte. Wichtig ist, dass man am inneren Rand der Platte misst und nicht am äusseren, da der äussere Rand ein Bisschen dicker ist. Jetzt kann der Wert direkt abgelesen und notiert werden. 

\section{Das optische System ausrichten}\label{sec:optischesSystem}
\subsection{Das Betrachtungsfernrohr fokussieren}
Die Betrachtungskammer sollte jetzt wieder zusammengebaut werden, das Gehäuse aber noch nicht. Auf der Platte soll der Fokussierdraht abgeschraubt werden und Vorsichtig in das Loch in der Mitte der oberen Kondensatorenplatte eingeführt werden. Danach muss die Halogenlampe angeschlossen werden. Dafür muss der Stecker des 12 V DC Transformator an die Lampe angeschlossen werden, dann sollte die Lampe leuchten. Jetzt muss zuerst das Fadenkreuz in Fokus gesetzt werden. Dafür muss man den Fadenkreuz-Fokussierring drehen bis man das komplette Gitter scharf sieht. Dann sollte man den Draht anschauen durch das Betrachtungsfernrohr und den Tröpfchen-Fokussierring solange drehen bis man den Draht scharf sehen kann. 

\subsection{Die Halogenlampe einstellen}\label{sub:Halogenlampe}
Mit dem horizontalen Einstellknopf der Halogenlampe soll das Licht auf der horizontalen Ebene richtig fokussiert werden. Damit das Licht am besten fokussiert ist, muss der rechte Rand des Drahtes am hellsten sein. Das heisst, im grössten Kontrast zur linken Seite des Drahtes stehen. Mit dem vertikalen Einstellungsknopf muss das Licht auf der Mitte des Gitters / Fadenkreuz am besten zu sehen sein.
	\chapter{Auswertung}\label{cha:auswertung}
Dieses Kapitel befasst sich mit der Genauigkeit der Bestimmung der Elementarladung. Zunächst wird ein Überblick auf die im Experiment erhaltenen Daten gegeben. In \autoref{tab:ergebnisse} sind die relevanten Messgrössen aufgeführt, nämlich die Steig- und Fallgeschwindigkeiten, der Radius, die Masse sowie die Ladung der Tröpfchen. Eine vollständige Übersicht der \textbf{Messwerte ist im Anhang zu finden}, wobei auch Messungen, die nicht ausreichend präzise waren, berücksichtigt werden. Diese wurden jedoch nicht in den Berechnungen einbezogen.

In der Fehlerrechnung (\autoref{sub:fehler}) wurde die Maturaarbeit von Herr Brenner \parencite{maturaarbeitBrenner} zur Hilfestellung der Funktionsweise einer Fehlerrechnung dazugezogen.

\section{Ausgehobene Daten}\label{sec:aushebungDaten}
Die Daten können nun in einem Punktdiagramm eingefügt werden, wobei die Y-Achse die Ladung der Tröpfchen und die X-Achse die jeweilige Nummer der Messung darstellt. Das Diagramm wurde mit Microsoft Excel erstellt und wird im Folgendem dargestellt.

\begin{figure}[h]
	\centering
	\includegraphics[width=\textwidth]{bilder/pdf/LadungsdiagrammOhne.pdf}
	\caption{Ladungsdiagramm ohne Fehlerrechnung}
	\label{fig:ladungsdiagrammOFehlerrechnung}
\end{figure}

\noindent Die Abstände auf der Y-Achse sind nicht zufällig gewählt, sondern entsprechen exakt einer Elementarladung. Anhand dieses Diagramms kann die Anzahl der Elementarladungen jedes einzelnen Tröpfchens abgelesen werden. Es war überraschend, wie präzise das Experiment verlief. Die Ladungen der Tröpfchen ordnen sich klar in Stufen, was während des Experimentierens nicht zu erwarten war. Weitere Details zur Genauigkeit der Messungen werden in \autoref{sec:genauigkeitAuswertung} behandelt, indem eine Fehlerrechnung zur Bewertung der Resultate gemacht wird. 

\section{Das Ergebnis}\label{sec:ergebnis}
In \autoref{tab:ergebnisse} kann nun eine weitere Spalte eingefügt werden, die die Anzahl der Elementarladungen $n$ angibt. Nachdem diese hinzugefügt wurde, sieht die Tabelle für die ersten 10 Zeilen wie folgt aus:

\begin{table}[H]
	\centering
	\begin{tabular}{llllll|l}
		\toprule
		Nr. & $v_{rise}$ [$m/s$] & $v_{fall}$ [$m/s$] & Radius [$m$] & Masse [$kg$]  & Ladung [$C$]& Anzahl (n) \\
		\midrule
		1 &$\mathrm{2.01 \cdot 10^{-04}}$ & $\mathrm{2.12 \cdot 10^{-05}}$ & $\mathrm{4.04 \cdot 10^{-07}}$ & $\mathrm{2.45 \cdot 10^{-16}}$ & $\mathrm{3.27 \cdot 10^{-19}}$ & 2\\
		2 &$\mathrm{8.46 \cdot 10^{-05}}$ & $\mathrm{2.16 \cdot 10^{-05}}$ & $\mathrm{4.08 \cdot 10^{-07}}$ & $\mathrm{2.53 \cdot 10^{-16}}$ & $\mathrm{1.59 \cdot 10^{-19}}$ & 1\\
		3 &$\mathrm{8.68 \cdot 10^{-05}}$ & $\mathrm{1.98 \cdot 10^{-05}}$ & $\mathrm{3.90 \cdot 10^{-07}}$ & $\mathrm{2.20 \cdot 10^{-16}}$ & $\mathrm{1.51 \cdot 10^{-19}}$ & 1\\
		5 &$\mathrm{8.91 \cdot 10^{-05}}$ & $\mathrm{2.04 \cdot 10^{-05}}$ & $\mathrm{3.96 \cdot 10^{-07}}$ & $\mathrm{2.31 \cdot 10^{-16}}$ & $\mathrm{1.58 \cdot 10^{-19}}$ & 1\\
		6 &$\mathrm{1.97 \cdot 10^{-04}}$ & $\mathrm{1.98 \cdot 10^{-05}}$ & $\mathrm{3.89 \cdot 10^{-07}}$ & $\mathrm{2.18 \cdot 10^{-16}}$ & $\mathrm{3.05 \cdot 10^{-19}}$ & 2\\
		7 &$\mathrm{1.98 \cdot 10^{-04}}$ & $\mathrm{2.04 \cdot 10^{-05}}$ & $\mathrm{3.96 \cdot 10^{-07}}$ & $\mathrm{2.31 \cdot 10^{-16}}$ & $\mathrm{3.14 \cdot 10^{-19}}$ & 2\\
		8 &$\mathrm{9.33 \cdot 10^{-05}}$ & $\mathrm{1.84 \cdot 10^{-05}}$ & $\mathrm{3.74 \cdot 10^{-07}}$ & $\mathrm{1.95 \cdot 10^{-16}}$ & $\mathrm{1.50 \cdot 10^{-19}}$ & 1\\
		9 &$\mathrm{1.26 \cdot 10^{-04}}$ & $\mathrm{1.71 \cdot 10^{-05}}$ & $\mathrm{3.61 \cdot 10^{-07}}$ & $\mathrm{1.74 \cdot 10^{-16}}$ & $\mathrm{1.85 \cdot 10^{-19}}$ & 1\\
		11 &$\mathrm{1.12 \cdot 10^{-04}}$ & $\mathrm{1.23 \cdot 10^{-05}}$ & $\mathrm{3.01 \cdot 10^{-07}}$ & $\mathrm{1.01 \cdot 10^{-16}}$ & $\mathrm{1.29 \cdot 10^{-19}}$ & 1\\
		12 &$\mathrm{1.30 \cdot 10^{-04}}$ & $\mathrm{1.29 \cdot 10^{-05}}$ & $\mathrm{3.08 \cdot 10^{-07}}$ & $\mathrm{1.09 \cdot 10^{-16}}$ & $\mathrm{1.53 \cdot 10^{-19}}$ & 1\\
		\bottomrule
		&&&&& $2.031 \cdot 10^{-18}$ & 13 \\
	\end{tabular}
	\caption{Ergebnisse mit Anzahl Ladungen}
	\label{tab:anzahlLadung}
\end{table}
\par
\noindent Die Anzahl der Elementarladungen wird addiert, was zu einer Summe von 13 Elementarladungen führt. Anschliessend werden die Ladungen summiert, was den Wert $2.031 \cdot 10^{-18}$ Coulomb ergibt. Das arithmetische Mittel dieser beiden Messwerten lautet dann: $\frac{2.031 \cdot 10^{-18}}{13} \ = \ 1.56 \cdot 10^{-19} Coulomb$. 

Wenn dieser Vorgang für alle Werte in der Tabelle wiederholt wird, ergibt sich eine Anzahl von 60 Elementarladungen und eine Gesamtsumme von $9.313 \cdot 10^{-18}$ Coulomb. Das Mittel dieser Werte führt zu einem Ergebnis für die Elementarladung von:

\begin{equation}\label{eq:ergebnis}
	\frac{9.313 \cdot 10^{-18}\ C}{60} \ = \ 1.552 \cdot 10^{-19}\ C
\end{equation}

\section{Die Genauigkeit}\label{sec:genauigkeitAuswertung}
Die Genauigkeit eines solchen Experimentes kann nur durch eine gründliche Fehlerrechnung bewertet werden. Im Folgenden wird die Fehlerrechnung erläutert, um den Einfluss der Fehlerquellen auf das Ergebnis hervorzuheben. Für jede relevante Grösse wird sowohl der absolute als auch der relative Fehler berechnet. Diese Fehler werden anschliessend miteinander kombiniert, um die Gesamtunsicherheit der Messergebnisse zu ermitteln.

In jedem Experiment treten Fehler auf, die entweder durch die Messinstrumente oder durch den Experimentieraufbau verursacht werden. Ein Beispiel für solche systematischen Fehler ist die mögliche Fehlkalibrierung des Multimeters oder eine ungenaue Erstellung des Fadenkreuz, bei dem die Gitternetzlinien nicht exakt 0,5 mm voneinander entfernt sind. Systematische Fehler können durch präzise Kalibrierung der Geräte und sorgfältige Vorbereitung des Experiments minimiert werden.

Zusätzlich gibt es auch die zufälligen Fehler, die durch unkontrollierbare Schwankungen während des Experiments auftreten. Beispiele hierfür sind die Reaktionszeit beim Starten der Stoppuhr, der Blickwinkel beim Beobachten der Tröpfchen oder auch Faktoren wie Temperatur- und Luftdruckschwankungen während des Experiments. Diese Fehler sind schwieriger zu eliminieren, können jedoch durch wiederholte Messungen und der Berechnung von Mittelwerten verringert werden, wodurch die zufälligen Fehler minimiert werden.

\subsection{Fehlerrechnung}\label{sub:fehler}
Zuerst müssen für alle Messgrössen die absoluten Fehler festgelegt werden. Diese können entweder geschätzt werden oder hängen von der Genauigkeit des jeweiligen Messgeräts ab.

\begin{table}[h]
	\begin{center}
		\begin{tabular}{l|lll}
			                             & Messwert     & absoluter Fehler     & relativer Fehler [\%] \\ \toprule
			Elektrisches Feld $[V/m]$    & zsmg. Grösse &                      & 0.68\%                \\
			Steigzeit $[s]$              & 3.7          & 0.1                  & 2.70\%                \\
			Sinkzeit $[s]$               & 32.6         & 0.1                  & 0.31\%                \\
			Distanz $[m]$                & 0.0005       & 0.00001              & 2.00\%                \\
			Steiggeschwindigkeit $[m/s]$ & zsmg. Grösse &                      & 4.70\%                \\
			Sinkgeschwindigkeit $[m/s]$  & zsmg. Grösse &                      & 2.31\%                \\
			Luftdruck $[Pa]$             & 94000        & 1000                 & 1.06\%                \\
			Zähigkeit $[Ns/m^2]$         & 0.00001818   & 0.00000001           & 0.05\%                \\
			Dichte $[kg/m^3]$            & 886          & 1                    & 0.11\%                \\
			Fallbeschleunigung $[m/s^2]$ & 9.81         & 0.05 & 0.51\%                \\
			Konstante b                  & 0.0082       & 0.00001              & 0.12\%
		\end{tabular}
	\end{center}
	\caption{Fehlertabelle Messgrössen}
	\label{tab:messabsFehler}
\end{table}

\noindent Zuerst wird der Fehler für den Radius berechnet:
\begin{equation*}\label{eq:fehlerRadius}
	\begin{split}
		Fehler_{Radius} & \ = \ \sqrt{\left( \frac{b}{2p}\right)^2 + \frac{9\eta v_f}{2\rho g}} - \frac{b}{2p} \ = \ \sqrt{\frac{9\eta v_f}{2\rho g}} - \frac{b}{2p} \\
		                & \ = \ \sqrt{\frac{0.05\% \cdot 2.31\%}{0.11\% \cdot 0.51\%}} - \frac{0.12\%}{1.06\%} \ = \ 1.49\% - 1.18\%                                              \\
		                & \ = \ 2.67\%
	\end{split}
\end{equation*}

\noindent Anschliessend kommt der Fehler für die Masse:
\begin{equation*}\label{eq:fehlerMasse}
	Fehler_{Masse} \ = \ \frac{4}{3} \pi \cdot a^3 \cdot \rho \ = \ (2.67\%)^3 \cdot 0.11\% \ = \ 8.12\%
\end{equation*}

\noindent Zuletzt erfolgt die Fehlerrechnung für die Ladung:
\begin{equation*}
	\begin{split}
		Fehler_{Ladung} & \ = \ \frac{mg(v_f + v_r)}{Ev_f} \ = \ \frac{8.12\% \cdot (4.70\% + 2.31\%)}{0.68\% \cdot 2.31\%} \\
		& \ = \ \frac{11.63\%}{2.99\%} \ = \ 14.62\%	
	\end{split}
\end{equation*}

\noindent Dieser Fehler für den Radius kann nun im Ladungsdiagramm berücksichtigt werden. Das Diagramm sieht dann wie folgt aus.

\begin{figure}[h]
	\centering
	\includegraphics[width=\textwidth]{bilder/pdf/LadungsdiagrammMitNeu.pdf}
	\caption{Ladungsdiagramm mit Fehlerrechnung}
	\label{fig:ladungsdiagrammMFehlerrechnung}
\end{figure}

\subsection{Schlussfolgerung des Ergebnis}\label{sub:schlussfolgerung}
Während des gesamten Experiments traten keine unplausiblen Werte auf, die eine deutlich zu hohe oder niedrige Ladung anzeigten. 

Mit diesem Schritt wird das Ergebnis des Millikan-Versuchs vollständig abgeschlossen. Das Diagramm, das zusammen mit den Fehlerberechnungen erstellt wurde, zeigt deutlich, dass alle Tröpfchen einer bestimmten Ladungsstufe zugeordnet werden können. Dies war während des Experiments zunächst nicht abzusehen. Erstaunlich ist, dass trotz der Vielzahl an Messgrössen, die jeweils mit unterschiedlichen Fehlerquellen behaftet sind, der berechnete Wert mit 19 Dezimalstellen konstant bleibt und im Bereich der theoretisch erwarteten Elementarladung liegt. Dies verdeutlicht die hohe Präzision des Experiments und die Stabilität des Ergebnisses, selbst bei der Auswertung der zahlreichen Messungen. \\

\noindent Der Fehler für das Messergebnis lautet wie folgt.

$$
\frac{\Delta Fehler}{Messwert} \ = \ rel. Fehler \ = \ \frac{\pm \ 5.002 \cdot 10^{-21}\ C}{1.552 \cdot 10^{-19}\ C} \ = \ 3.22 \%
$$

\noindent Wie in dieser Rechnung gezeigt, liegt der relative Fehler des Messergebnis im Rahmen des Fehlers vom Experiment. 





\begin{table}
\caption{Ergebnisse der Berechnung}
\label{tab:ergebnisse}
\centering
\begin{tabular}{lllll}
\toprule
$v_{rise}$ & $v_{fall}$ & $Radius$ & $Masse$ & $Ladung$ \\
\midrule
$\mathrm{2.01 \cdot 10^{-04}}$ & $\mathrm{2.12 \cdot 10^{-05}}$ & $\mathrm{4.04 \cdot 10^{-07}}$ & $\mathrm{2.45 \cdot 10^{-16}}$ & $\mathrm{3.27 \cdot 10^{-19}}$ \\
$\mathrm{8.46 \cdot 10^{-05}}$ & $\mathrm{2.16 \cdot 10^{-05}}$ & $\mathrm{4.08 \cdot 10^{-07}}$ & $\mathrm{2.53 \cdot 10^{-16}}$ & $\mathrm{1.59 \cdot 10^{-19}}$ \\
$\mathrm{8.68 \cdot 10^{-05}}$ & $\mathrm{1.98 \cdot 10^{-05}}$ & $\mathrm{3.90 \cdot 10^{-07}}$ & $\mathrm{2.20 \cdot 10^{-16}}$ & $\mathrm{1.51 \cdot 10^{-19}}$ \\
$\mathrm{8.91 \cdot 10^{-05}}$ & $\mathrm{2.04 \cdot 10^{-05}}$ & $\mathrm{3.96 \cdot 10^{-07}}$ & $\mathrm{2.31 \cdot 10^{-16}}$ & $\mathrm{1.58 \cdot 10^{-19}}$ \\
$\mathrm{1.97 \cdot 10^{-04}}$ & $\mathrm{1.98 \cdot 10^{-05}}$ & $\mathrm{3.89 \cdot 10^{-07}}$ & $\mathrm{2.18 \cdot 10^{-16}}$ & $\mathrm{3.05 \cdot 10^{-19}}$ \\
$\mathrm{1.98 \cdot 10^{-04}}$ & $\mathrm{2.04 \cdot 10^{-05}}$ & $\mathrm{3.96 \cdot 10^{-07}}$ & $\mathrm{2.31 \cdot 10^{-16}}$ & $\mathrm{3.14 \cdot 10^{-19}}$ \\
$\mathrm{9.33 \cdot 10^{-05}}$ & $\mathrm{1.84 \cdot 10^{-05}}$ & $\mathrm{3.74 \cdot 10^{-07}}$ & $\mathrm{1.95 \cdot 10^{-16}}$ & $\mathrm{1.50 \cdot 10^{-19}}$ \\
$\mathrm{1.26 \cdot 10^{-04}}$ & $\mathrm{1.71 \cdot 10^{-05}}$ & $\mathrm{3.61 \cdot 10^{-07}}$ & $\mathrm{1.74 \cdot 10^{-16}}$ & $\mathrm{1.85 \cdot 10^{-19}}$ \\
$\mathrm{1.12 \cdot 10^{-04}}$ & $\mathrm{1.23 \cdot 10^{-05}}$ & $\mathrm{3.01 \cdot 10^{-07}}$ & $\mathrm{1.01 \cdot 10^{-16}}$ & $\mathrm{1.29 \cdot 10^{-19}}$ \\
$\mathrm{1.30 \cdot 10^{-04}}$ & $\mathrm{1.29 \cdot 10^{-05}}$ & $\mathrm{3.08 \cdot 10^{-07}}$ & $\mathrm{1.09 \cdot 10^{-16}}$ & $\mathrm{1.53 \cdot 10^{-19}}$ \\
$\mathrm{1.26 \cdot 10^{-04}}$ & $\mathrm{1.15 \cdot 10^{-05}}$ & $\mathrm{2.89 \cdot 10^{-07}}$ & $\mathrm{8.97 \cdot 10^{-17}}$ & $\mathrm{1.37 \cdot 10^{-19}}$ \\
$\mathrm{1.21 \cdot 10^{-04}}$ & $\mathrm{1.58 \cdot 10^{-05}}$ & $\mathrm{3.45 \cdot 10^{-07}}$ & $\mathrm{1.52 \cdot 10^{-16}}$ & $\mathrm{1.68 \cdot 10^{-19}}$ \\
$\mathrm{2.26 \cdot 10^{-04}}$ & $\mathrm{1.81 \cdot 10^{-05}}$ & $\mathrm{3.72 \cdot 10^{-07}}$ & $\mathrm{1.91 \cdot 10^{-16}}$ & $\mathrm{3.31 \cdot 10^{-19}}$ \\
$\mathrm{4.50 \cdot 10^{-04}}$ & $\mathrm{1.21 \cdot 10^{-05}}$ & $\mathrm{2.97 \cdot 10^{-07}}$ & $\mathrm{9.76 \cdot 10^{-17}}$ & $\mathrm{4.79 \cdot 10^{-19}}$ \\
$\mathrm{2.84 \cdot 10^{-04}}$ & $\mathrm{1.07 \cdot 10^{-05}}$ & $\mathrm{2.76 \cdot 10^{-07}}$ & $\mathrm{7.83 \cdot 10^{-17}}$ & $\mathrm{2.79 \cdot 10^{-19}}$ \\
$\mathrm{2.43 \cdot 10^{-04}}$ & $\mathrm{1.40 \cdot 10^{-05}}$ & $\mathrm{3.23 \cdot 10^{-07}}$ & $\mathrm{1.25 \cdot 10^{-16}}$ & $\mathrm{2.93 \cdot 10^{-19}}$ \\
$\mathrm{1.26 \cdot 10^{-04}}$ & $\mathrm{1.28 \cdot 10^{-05}}$ & $\mathrm{3.07 \cdot 10^{-07}}$ & $\mathrm{1.07 \cdot 10^{-16}}$ & $\mathrm{1.48 \cdot 10^{-19}}$ \\
$\mathrm{3.97 \cdot 10^{-04}}$ & $\mathrm{1.30 \cdot 10^{-05}}$ & $\mathrm{3.10 \cdot 10^{-07}}$ & $\mathrm{1.11 \cdot 10^{-16}}$ & $\mathrm{4.44 \cdot 10^{-19}}$ \\
$\mathrm{2.49 \cdot 10^{-04}}$ & $\mathrm{1.45 \cdot 10^{-05}}$ & $\mathrm{3.29 \cdot 10^{-07}}$ & $\mathrm{1.32 \cdot 10^{-16}}$ & $\mathrm{3.06 \cdot 10^{-19}}$ \\
$\mathrm{3.96 \cdot 10^{-05}}$ & $\mathrm{3.36 \cdot 10^{-05}}$ & $\mathrm{5.20 \cdot 10^{-07}}$ & $\mathrm{5.22 \cdot 10^{-16}}$ & $\mathrm{1.46 \cdot 10^{-19}}$ \\
$\mathrm{1.04 \cdot 10^{-04}}$ & $\mathrm{1.47 \cdot 10^{-05}}$ & $\mathrm{3.31 \cdot 10^{-07}}$ & $\mathrm{1.35 \cdot 10^{-16}}$ & $\mathrm{1.40 \cdot 10^{-19}}$ \\
$\mathrm{5.22 \cdot 10^{-05}}$ & $\mathrm{2.89 \cdot 10^{-05}}$ & $\mathrm{4.79 \cdot 10^{-07}}$ & $\mathrm{4.08 \cdot 10^{-16}}$ & $\mathrm{1.47 \cdot 10^{-19}}$ \\
$\mathrm{5.40 \cdot 10^{-05}}$ & $\mathrm{3.06 \cdot 10^{-05}}$ & $\mathrm{4.95 \cdot 10^{-07}}$ & $\mathrm{4.50 \cdot 10^{-16}}$ & $\mathrm{1.59 \cdot 10^{-19}}$ \\
$\mathrm{5.82 \cdot 10^{-05}}$ & $\mathrm{2.67 \cdot 10^{-05}}$ & $\mathrm{4.59 \cdot 10^{-07}}$ & $\mathrm{3.60 \cdot 10^{-16}}$ & $\mathrm{1.46 \cdot 10^{-19}}$ \\
$\mathrm{5.34 \cdot 10^{-05}}$ & $\mathrm{2.98 \cdot 10^{-05}}$ & $\mathrm{4.88 \cdot 10^{-07}}$ & $\mathrm{4.30 \cdot 10^{-16}}$ & $\mathrm{1.54 \cdot 10^{-19}}$ \\
$\mathrm{5.76 \cdot 10^{-05}}$ & $\mathrm{2.79 \cdot 10^{-05}}$ & $\mathrm{4.71 \cdot 10^{-07}}$ & $\mathrm{3.87 \cdot 10^{-16}}$ & $\mathrm{1.52 \cdot 10^{-19}}$ \\
$\mathrm{5.81 \cdot 10^{-05}}$ & $\mathrm{3.01 \cdot 10^{-05}}$ & $\mathrm{4.91 \cdot 10^{-07}}$ & $\mathrm{4.38 \cdot 10^{-16}}$ & $\mathrm{1.64 \cdot 10^{-19}}$ \\
$\mathrm{5.32 \cdot 10^{-04}}$ & $\mathrm{1.89 \cdot 10^{-05}}$ & $\mathrm{3.81 \cdot 10^{-07}}$ & $\mathrm{2.06 \cdot 10^{-16}}$ & $\mathrm{7.67 \cdot 10^{-19}}$ \\
$\mathrm{3.55 \cdot 10^{-04}}$ & $\mathrm{1.90 \cdot 10^{-05}}$ & $\mathrm{3.82 \cdot 10^{-07}}$ & $\mathrm{2.06 \cdot 10^{-16}}$ & $\mathrm{5.21 \cdot 10^{-19}}$ \\
$\mathrm{3.31 \cdot 10^{-04}}$ & $\mathrm{1.65 \cdot 10^{-05}}$ & $\mathrm{3.54 \cdot 10^{-07}}$ & $\mathrm{1.64 \cdot 10^{-16}}$ & $\mathrm{4.42 \cdot 10^{-19}}$ \\
$\mathrm{4.85 \cdot 10^{-04}}$ & $\mathrm{1.78 \cdot 10^{-05}}$ & $\mathrm{3.68 \cdot 10^{-07}}$ & $\mathrm{1.85 \cdot 10^{-16}}$ & $\mathrm{6.71 \cdot 10^{-19}}$ \\
$\mathrm{2.10 \cdot 10^{-04}}$ & $\mathrm{1.66 \cdot 10^{-05}}$ & $\mathrm{3.55 \cdot 10^{-07}}$ & $\mathrm{1.66 \cdot 10^{-16}}$ & $\mathrm{2.89 \cdot 10^{-19}}$ \\
$\mathrm{1.04 \cdot 10^{-04}}$ & $\mathrm{1.77 \cdot 10^{-05}}$ & $\mathrm{3.67 \cdot 10^{-07}}$ & $\mathrm{1.83 \cdot 10^{-16}}$ & $\mathrm{1.61 \cdot 10^{-19}}$ \\
$\mathrm{2.14 \cdot 10^{-04}}$ & $\mathrm{7.96 \cdot 10^{-06}}$ & $\mathrm{2.34 \cdot 10^{-07}}$ & $\mathrm{4.75 \cdot 10^{-17}}$ & $\mathrm{1.69 \cdot 10^{-19}}$ \\
$\mathrm{6.02 \cdot 10^{-04}}$ & $\mathrm{8.06 \cdot 10^{-06}}$ & $\mathrm{2.36 \cdot 10^{-07}}$ & $\mathrm{4.85 \cdot 10^{-17}}$ & $\mathrm{4.70 \cdot 10^{-19}}$ \\
\bottomrule
\end{tabular}
\end{table}

	\chapter{Fazit}\label{cha:fazit}
Die ursprüngliche Frage dieser Arbeit lautete, wie genau wird das Ergebnis einer Elementarladung ausfällt, wenn das Experiment, das vor über hundert Jahren entwickelt wurde, heute wiederholt wird. Die Antwort darauf ist, verblüffend genau. Mithilfe eines Experimentierkasten wurde dieses Experiment durchgeführt und zunächst war das Ergebnis ernüchternd. Es gab vieles zu beachten, wobei die sorgfältige Berechnung der Ladung am schwierigsten war. Die Berechnung hängt von 12 verschiedenen Grössen ab, die alle unterschiedliche Einheiten und verschiedene Grössenordnungen besitzen. Wenn hier nicht konzentriert gearbeitet wurde, war das Endergebnis am fehlerhaft. Eine weitere Herausforderung war das Messen. Wo genau befanden sich die 0.5mm Linien? War genug Spannung vorhanden? Wurde die Luftviskosität korrekt abgelesen? All diese Faktoren erchwerten zu Beginn das Erreichen eines brauchbaren Resultates. Mit der Zeit konnten jedoch diese Schwierigkeiten überwunden werden und schliesslich wurden Ergebnisse erzielt, die in der Grössenordnung von $10^{-19}$ lagen. Die Zuversicht wuchs und man es war erstaunlich, das Endergebnis in den Händen zu halten. 

\section{Methoden}\label{sec:methoden}
Dieses Experiment erfordert viel Geduld und Präzision. Das grösste Problem lag an den Berechnungen. Eine komplexe Formel und gleichzeitig etwa 40 verschiedene Messungen, die verarbeitet werden mussten. Wie kann dieser Prozess vereinfacht und währenddessen Zeit effizienter genutzt werden? Die Antwort darauf liegt in der Programmierung. Für Berechnungen der Ergebnisse wurde Python verwendet. Mit dem Pandas-Modul konnten Daten aus Excel-Tabellen gelesen, verarbeitet und wieder zurückgeschrieben werden \parencite[vgl.]{Inc_2024}. 
Da diese Arbeit in \LaTeX~geschrieben wurde, mussten alle Grafiken von .png in .pdf Format konvertiert werden. Auch diesen Schritt übernahm ein Python-Skript. 
Im Allgemeinen konnte ich durch diese Arbeit meine Programmierkenntnisse vertiefen. Das Arbeiten mit \LaTeX~war anfangs herausfordernd, aber es hat sich gelohnt, da es das Schreiben von Formeln vereinfacht hat und das Formatieren einer wissenschaftlichen Arbeit ebenfalls. 

\section{Resümee}\label{sec:resumee}
Diese Arbeit hat gezeigt, wie es ist ein fortgeschritteneres Experiment im Fach Physik durchzuführen. Man hat gelernt, wie Messreihen aufgestellt werden, wie ein Experiment sorgfältig geplant werden muss und dass man nicht aufgeben sollte, wenn das gewünschte Ergebnis nicht sofort eintritt. Das Experiment wird nicht beim ersten Versuch gelingen. Es erfordert Übung, Wissen und Planung, andernfalls wird es nicht gelingen. Eine weitere wichtige Erkenntnis war, dass es nicht schadet sich Unterstützung zu holen bei Fachpersonen und Mitschüler:innen. Das Experimentieren gelang erst als ein Assistenten hinzugezogen wurde. \\

\noindent Die vorliegende Arbeit hat zudem gezeigt, dass eine sorgfältige Arbeitsweise sich auszahlen kann und dass die Inanspruchnahme externer Hilfe bei Schwierigkeiten zu einem erfolgreichen Ergebnis führen kann.

\begin{figure}[h]
	\centering
	\includegraphics[scale=0.25]{bilder/pdf/bildExperimentieren.pdf}
	\caption{Samuel während dem Experimentieren}
	\label{fig:experimentieren}
\end{figure}
	\chapter*{Eigenständigkeitserklärung}\label{cha:eigenständigkeitserklärung}
Ich bestätige mit meiner Unterschrift, dass ich meine Maturaarbeit selbständig verfasst und in schriftliche Form gebracht habe, dass sich die Mitwirkung anderer Personen auf Beratung und Korrekturlesen beschränkt hat und dass alle verwendeten Unterlagen und Gewährspersonen aufgeführt sind. Mir ist bekannt, dass eine Maturaarbeit, die nachweislich ein Plagiat gemäss Art. 1quater des Maturitätsprüfungsreglements des Gymnasiums (s. auch Maturaarbeitsbroschüre) darstellt, als schwerer Verstoss gewertet wird.
\vspace{1.5cm}

\noindent Datum und Unterschrift: \rule[-2mm]{11cm}{0.4pt} 
	
	
	
	
	
	\cleardoublepage
	\listoffigures
	\listoftables
	\printbibliography
	
\end{document}





