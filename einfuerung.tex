\chapter{Einführung}\label{ch:einfuerung}
\section{Einleitung}\label{sec:einleitung}
Warum wird ein Draht heiss wenn Strom durchfliesst? Aus dem gleichen Grund können wir unser Handy aufladen. Die Elektronen sind das, was wir heute unter Strom verstehen.  
Stellen Sie sich vor, Sie untersuchen die kleinsten geladenen Teilchen, die man bisher kennt. Sie erhalten sehr genaue Messwerte und glauben, die Natur der Elementarladung erforscht zu haben. Sie veröffentlichen Ihre Arbeit, aber eifersüchtige Konkurrenten versuchen Sie niederzumachen, indem sie behaupten, Sie hätten Messwerte ausgeschlossen, die nicht stimmten. Sie versuchen, diesen Konkurrenten zu beweisen, dass das, was Sie erforscht haben, wahr ist. Es vergeht eine lange Zeit, in der Sie sich beweisen müssen, bis Sie etwa 13 Jahre später mit Ihrer Entdeckung der Elementarladung den Nobelpreis für Physik gewinnen. Sie haben genau das durchgemacht, was Robert Andrews {\scshape Millikan} in den Jahren des Ersten Weltkrieges durchgemacht hat.

Diese Arbeit soll zeigen, wie man zu Beginn des 20. Jahrhunderts auf ein solches Experiment gekommen ist, wie man vor mehr als hundert Jahren so genaue Messwerte erhalten hat und wie genau solche Messungen sind, wenn man das Experiment heute wiederholt.

% Quelle: https://de.wikipedia.org/wiki/Robert_Andrews_Millikan

\section{Biographie von Millikan}\label{sec:autobiographie}
Robert Andrews {\sc Millikan} wurde 1868 in Amerika geboren. Im Alter von 18 Jahren begann er am Oberlin College (Ohio) zu studieren. Zunächst studierte er Mathematik und Griechisch, später belegte er einen Kurs in Physik und legte sein Examen als Physiklehrer ab. Etwa 10 Jahre später promovierte er an der Columbia University. Nach seiner Promotion ging er für ein Jahr nach Deutschland, um seine Kenntnisse bei Max {\sc Planck} und Walther {\sc Nernst} zu vertiefen. Danach kehrte er in die USA zurück, wo er 10 Jahre als Professor an der University of Chicago arbeitete. 

1909 begann er, die Natur der Elementarladung zu erforschen. Anfangs benutzte er die \textit{Tröpfchenmethode}, die mit Wasser durchgeführt wurde. Später benutzte er die \textit{Öltröpfchenmethode}, die für die Bestimmung der Elementarladung besser geeignet war, da sich Öltröpfchen im Vergleich zu Wassertröpfchen als stabiler erwiesen.  Mit dieser Methode gelang es ihm, die Einheit der kleinsten elektrischen Ladung zu bestimmen, die er mit "e" bezeichnete. Ein Jahr später veröffentlichte er seine Arbeit mit mehr als 38 Messungen. Sie stieß bei anderen Forschern auf großes Interesse, aber auch auf heftige Kritik. Um die Kritik zu entkräften, veröffentlichte er drei Jahre später eine weitere Arbeit über die experimentelle Bestimmung der Elementarladung, doch auch diese Ergebnisse wurden angezweifelt. In den Jahren vor dem Ersten Weltkrieg erhielt er 3-4 Auszeichnungen, darunter den Comstock-Preis für Physik.

Millikan untersuchte nicht nur die Natur der Elementarladung, sondern wollte auch die Lichtquantenhypothese von Albert {\sc Einstein} experimentell überprüfen, da er Einsteins Interpretation skeptisch gegenüberstand. Es gelang ihm jedoch, die Richtigkeit von Einsteins Gleichungen zu beweisen.

Als Millikan 1918 sein Buch "Das Elektron" veröffentlichte, behauptete er, seine Messungen der Elementarladung seien genauer als die seiner Konkurrenten, da die Werte nur wenig streuten. Diese Arbeit begründete seinen späteren Ruhm und die Verleihung des Nobelpreises im Jahr 1923. 

In der Zwischenkriegszeit setzte er seine Forschungen fort, bis er 1946 in den Ruhestand trat. Er schrieb zahlreiche Bücher über Natur und Religion sowie verschiedene Lehrbücher. \parencite[vgl.][Millikan]{dewiki:247237013}

\section{Relevanz der Elementarladung Heute}\label{sec:relevanz}
In welchen Bereichen des täglichen Lebens benötigen wir heute Elementarladungen? Die wohl bekannteste Technik, bei der wir reine Elementarladungen (Elektronen) benötigen, ist die Röntgentechnik in der Medizin. Ein Röntgengerät ist nichts anderes als ein Teilchenbeschleuniger, der Elektronen auf den bzw. durch den Körper schiesst. Ein anderes Beispiel aus der Medizin ist das Bestrahlungsgerät in der Krebstherapie. Hier werden keine Elektronen, sondern Protonen mit genau der gleichen Ladung, $1.602176634 \cdot 10^{-19} C$ \parencite[123]{fundamentum_mathe}, aber positiv statt negativ, beschleunigt und auf den Körper geschossen. 

Ohne das Wissen, dass es keine Ladung gibt, die kleiner als die Elementarladung ist, würde heute keines unserer elektronischen Geräte, insbesondere keine elektronischen Rechner, funktionieren. Denn jedes Bit in unseren Chips basiert darauf, ob ein Elektron fehlt oder nicht.

Um die Masse eines Elektrons zu bestimmen, benötigt man auch die Elementarladung \textit{e}. Mit Hilfe eines Magneten wird ein Elektron auf eine Kreisbahn geschickt. Dabei wirkt auf das Elektron eine magnetische Kraft (Lorenzkraft), die es auf eine Kreisbahn schickt. Um eine Kreisbewegung zu erzielen, braucht es eine Zentripetalkraft, die von der Lorenzkraft aufgebraucht wird. Formal ausgedrückt bedeutet das: $F_L \ = \ F_Z$, wobei $F_L \ = \ e \cdot v \cdot B$. Dabei steht e für die Elementarladung

   
